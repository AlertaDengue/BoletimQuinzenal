\section{Introdução}

Um número excepcional de casos de dengue foram registrados globalmente em 2024. O mesmo ocorreu no Brasil, com epidemias em quase todos os estados e taxas de ataque muito maiores do que historicamente conhecido. Observou-se também a consolidação da expansão da dengue para áreas ao sul e em altitudes onde não se registrava epidemias de dengue antes de 2020. Além disso, observou-se nesse ano, a co-circulação de chikungunya e febre oropouche, que por compartilharem sintomas iniciais semelhantes, podem ter contribuido para o aumento de casos notificados como dengue.

Mediante esse cenário, foi lançado em junho de 2024, o Sprint Infodengue com o objetivo de gerar projeções de casos de dengue para o ano de 2025 a partir da experiência de vários grupos de modelagem. Sabe-se que a combinação de modelos fortalece a capacidade preditiva, por meio da geração de ensembles. O sprint Infodengue foi conduzido pelas equipes do Infodengue e da plataforma Mosqlimate com a seguinte pergunta principal:

\textbf{\textit{Qual será o número esperado de casos prováveis de dengue na temporada 2024-25, no Brasil como um todo, e por Unidade de Federação (UF)?
}}
O objetivo principal do \textit{Dengue 2024 Sprint} foi promover o treinamento de modelos para a dengue no Brasil e fornecer previsões para 2025, enquanto os objetivos específicos foram: 

\begin{itemize}
    \item Organizar uma comunidade de modeladores com objetivos e métodos unificados 
    \item Gerar um conjunto de modelos independentes testados usando dados de temporadas anteriores 
    \item Treinar modelo de conjunto com todos os envios
    \item Produzir previsões para 2025 usando os melhores modelos, simples ou combinados 
    \item Atualizar e monitorar o desempenho dos modelos em 2025
\end{itemize}
 
Para o sprint, a plataforma Mosqlimate organizou uma base abrangente de dados epidemiológicos, demográficos e climáticos, assim como um template no Github para o treinamento de modelos preditivos visando a elaboração de um ensemble de previsões. 

\subsection{Participantes do Sprint}

%to do