\section{Destaques}
\tcbset{colback=blue!5!white}
\begin{tcolorbox}[enhanced,title=Importante]
    \lipsum{1}
\end{tcolorbox}

Em 2023, índicios de manutenção de circulação de dengue no inverno e a perspectiva de clima propício para transmissão de arboviroses, devido ao El Nino, levou o Infodengue a lançar um relatório de alerta sobre o risco de epidemia em 2024. Esse relatório serviu de base para organização do serviço e da resposta a nível nacional e estadual. Ainda assim, o cenário real superou as expectativas em termos de morbidade e mortalidade.

O objetivo desse relatório é apresentar previsões para a temporada de dengue 2024-2025, período de outubro de 2024 a setembro de 2025, a partir de uma coleção de modelos independentemente gerados por vários grupos de modelagem com experiência em dengue. Essa iniciativa, chamada de \textit{Dengue 2024 Sprint}, foi organizado pela plataforma Mosqlimate, em parceria com o Infodengue.     

A principal premissa do desafio é que prever onde e como a próxima epidemia ocorrerá é importante para alocar recursos para reduzir a carga de doenças, permitindo uma resposta oportuna. Assim, o \textit{Dengue 2024 Sprint foi } teve como produto final fornecer previsões para 2025, em nível estadual, por meio de uma comunidade de modeladores com metas e métodos unificados que geraram um conjunto de modelos independentes testados com dados de temporadas anteriores da doença. 

O Mosqlimate é um projeto de pesquisa cuja missão é gerar metodologias para facilitar o processo de modelagem do impacto do clima na dinâmica dos arboviroses, e é responsável pela construção e manutenção de uma plataforma padronizada para validação de modelos.  

