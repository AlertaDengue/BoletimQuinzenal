\section{Métodos}

\subsection{Dados}
% decricao das variaveis e suas fontes. 

A plataforma Mosqlimate trabalha sobre vários fluxos de dados com acesso por meio de endpoints disponiveis em https://api.mosqlimate.org/datastore/: 

 \textbf{Dados de Incidência.}Um endpoint dá acesso a dados do projeto Infodengue, que fornece uma série de variáveis epidemiológicas para todos os municípios brasileiros em uma escala de tempo semanal. 
 
\textbf{Dados  da expansão da dengue, zika e chikungunya no Brasil }sao obtidos do projeto Epi-Scanner, que recupera e analisa  informações dos dados de incidência atualizados do Infodengue  
 \textbf{P}\textbf{revisões} \textbf{climáticas}. Por meio deste endpoint da API, é possível buscar várias variáveis climáticas que foram extraídas para todos os municípios brasileiros em uma escala de tempo diária a partir dos dados de reanálise baseados em satélite fornecidos pelo Copernicus ERA5. Dados de temperatura e umidade são obtidos das \href{https://estacoes.dengue.mat.br/}{estações meteorológicas} de aeroportos assim como de imagens de satélite.

 \textbf{D}\textbf{ados} \textbf{de} \textbf{densidade} \textbf{populacional} \textbf{de} \textbf{mosquitos} (\textbf{Ovicounter}): Este endpoint tem como função acesso aos dados de abundância de mosquitos do projeto Contaovos, codesenvolvido pelo projeto Mosqlimate. Esses dados são baseados em armadilhas de ovos distribuídas por todo o Brasil de acordo com um design de monitoramento especificado pelo Ministério da Saúde. 

 \textbf{Dados demográficos:} Para o cálculo de indicadores epidemiológicos, dados demográficos dos municípios brasileiros são atualizados a cada ano no Infodengue, utilizando as estimativas do \href{https://www.ibge.gov.br/pt/inicio.html}{IBGE.}

As variáveis disponiveis para o conjunto de endpoints estao listadas a seguir: 

\begin{enumerate}
    \item Semana epidemiológica
    \item Número acumulado de casos no ano 
    \item Número estimado de casos por semana usando o modelo de nowcasting (nota: Os valores são atualizados retrospectivamente a cada semana)
    \item Intervalo de credibilidade de 95\% do número estimado de casos
    \item Número de casos notificados por semana (Os valores são atualizados retrospectivamente todas as semanas)
    \item Probabilidade de (Rt> 1). Para emitir o alerta laranja, usamos o critério p\_rt1> 0,95 por 3 semanas ou mais.
    \item Taxa de incidência estimada por 100.000
    \item Divisão submunicipal (atualmente implementada apenas no Rio de Janeiro)
    \item Nível de alerta (1 = verde, 2 = amarelo, 3 = laranja, 4 = vermelho)
    \item Índice numérico
    \item Estimativa pontual do número reprodutivo de casos (Rt)
    \item População estimada (IBGE)
    \item Média das temperaturas mínimas diárias ao longo da semana
    \item Média das temperaturas diárias ao longo da semana
    \item Média das temperaturas máximas diárias ao longo da semana
    \item Média da umidade relativa mínima diária do ar ao longo da semana
    \item Média da umidade relativa diária do ar ao longo da semana
    \item Média da umidade relativa máxima diária do ar ao longo da semana
    \item Receptividade climática, ou seja, condições para alta capacidade vetorial onde 
    \begin{enumerate}
        \item 0 = desfavorável
        \item 1 = favorável
        \item  2 = favorável nesta semana e na semana passada 
        \item 3 = favorável por pelo menos três semanas (suficiente para completar um ciclo de transmissão)
    \end{enumerate}
    \item Evidência de transmissão sustentada onde: 
    \begin{enumerate}
        \item 0 = nenhuma evidência
        \item 1 = possível
        \item  2 = provável
        \item 3 = altamente provável
    \end{enumerate}
    \item Incidência estimada abaixo do limiar pré-epidemia onde, 
    \begin{enumerate}
        \item 1 = acima do limiar pré-epidemia, mas abaixo do limiar epidêmico
    \end{enumerate}
    \begin{enumerate}
        \item 2 = acima do limiar epidêmico
    \end{enumerate}


\end{enumerate}


\subsection{Desafios}

% descrever os desafios (que periodos e UFs)
O desafio foi composto por duas etapas, dois testes de validação e uma meta de previsão. Os testes de validação e previsão consistiram em prever o número semanal de casos de dengue por estado (UF) nas temporadas 2022-2023 e 2023-2024 usando dados de casos notificados entre 2010 a 2023; enquanto a meta de previsão consistiu em prever o número semanal de casos de dengue no Brasil, e por estado (UF), na temporada 2024-2025 utilizando dados do período de 2010 a 2024. 

O período das previsões foi entre a semana epidemiológica (SE) 41 de um ano e a SE 40 do ano seguinte, que corresponde a uma temporada típica de dengue no Brasil. Os conjuntos de dados de treinamento e seu respectivo dicionário de variáveis foram disponibilizados na plataforma Mosqlimate. Em geral, o conjunto de dados continha dados epidemiológicos, demográficos e climáticos de acesso aberto, atualizáveis e disponíveis para todos os estados brasileiros.

Todas as previsões da equipe foram cadastradas na plataforma Mosqlimate, usando a seguinte ferramenta de registro https://api.mosqlimate.org/models/.  A plataforma, que esta projetada para comparar experimentos de previsão de arbovírus realiza as analises conforme metricas de avaliaçao definidas pela equipe. 
 


\subsection{Scores de avaliação dos modelos}

% nao precisa colocar formulas, podemos só explicar o que são, o que eles medem,  e citar o github. 
O desempenho de cada modelo foi avaliado utilizando um conjunto de pontuações e métricas que consideraram o início da epidemia e o pico da epidemia e o intervalo de tempo, maximizando a correlação cruzada entre previsões e dados. 

 Analisando os resultados de cada predição, foi calculado um score que indicava os melhores resultados gerados pelos modelos propostos para cada estado e cada ano, correspondente aos testes de validação e previsão. Para cada ano e estado, os modelos foram avaliados de acordo com a pontuação e a semana epidemiológica prevista e com base nessas pontuações, os modelos de concordância foram classificados com diferentes scores. Uma classificação global também foi calculada utilizando um método semelhante. 

A pontuação logarítmica, CRPS e pontuação de intervalo foram calculadas usando o pacote Python \textbackslash{}href\{https://github.com/Mosqlimate-project/mosqlimate-client/tree/main\}{mosqlient\}, que captura as previsões da API e as compara usando algumas pontuações implementadas no pacote Python scoringrules.

Finalmente, um modelo ensemble foi treinado para produzir previsões para 2025 utilizando os melhores modelos, únicos ou combinados. Para o ensemble, finalmente, os modelos foram adicionados incrementalmente, seguindo a ordem de classificação até que não houvessem mais melhoria no desempenho do modelo. 

 
\subsection{Modelos utilizados}

\subsection{Construção de ensembles}

\subsection{Forecast para 2025}




